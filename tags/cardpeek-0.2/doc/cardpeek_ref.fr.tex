\documentclass[11pt]{article} 
\usepackage{url} 
\title{Manuel \texttt{Cardpeek}\\{\huge Avertissement : Traduction en cours\\voir le texte en Anglais.}}
\author{Copyright 2009, L1L1@gmx.com} 
\addtolength{\textwidth}{2cm} 
\addtolength{\hoffset}{-1cm} 
\begin{document} 

\maketitle 

Ceci est le manuel de \texttt{Cardpeek}, un outil de lecture de carte \`a puce
sous GNU Linux.


\section{Pr\'esentation}

\texttt{Cardpeek} est un outil de lecture de carte \`a puce avec une interface graphique bas\'ee sur GTK 2.0, fonctionnant sous GNU Linux et extensible par le langage de programmation LUA. 
Ce logiciel n\'ecessite un lecteur PCSC pour fonctionner.

Les cartes \`a puce sont devenues banales dans notre quotidien. Nous les utilisons pour le paiement, le transport, le t\'el\'ephone et beaucoup d'autres applications. 
Ces cartes contiennent bien souvent beaucoup d'informations, comme par exemple les traces de nos derni\`eres transactions bancaires ou encore la trace de nos derniers d\'eplacements dans les transports publics. 

\texttt{Cardpeek} est un outil qui a pour objectif de vous permettre d'acc\'eder \`a ces informations personnelles. Vous pouvez ainsi \^etre mieux inform\'e des donn\'ees qui sont collect\'ees sur vous.

\texttt{Cardpeek} explore le contenu d'une carte \`a puce respectant les normes ISO 7816 et la repr\'esente sous la forme d'une arborescence en respectant grossi\`erement la structure qu'elles ont sur la carte.

Dans cette version, l'application est capable de lire le contenu des cartes suivantes :
\begin{itemize}
\item{Les cartes bancaires EMV}
\item{Les cartes de transport parisien Navigo et certaines autres cartes similaires utilis\'ees en France.}
\item{Les cartes Mon\'eo}
\end{itemize}

La lecture des cartes de transport est encore en version `b\'eta', et il manque d'autres types de cartes comme les cartes SIM des t\'el\'ephones mobiles. 
Heureusement, l'application peut \^etre modifi\'ee et entendue simplement gr\^ace au langage de script LUA. 
Pour plus d'informations sur le projet LUA, voir \texttt{http://www.lua.org/}.

\section{Installation}

Utiliser le \texttt{Makefile} pour construire le programme \`a partir de sa source. 
Ce logiciel n\'ecessite les librairies GTK, LUA et PCSCLITE.
Pour plus de d\'etails, lire le fichier \texttt{INSTALL}.

\section{Descriptif de l'interface utilisateur}

L'interface utilisateur est partag\'ee en trois parties : le menu, l'arborescence et la console de messages.

Le menu contient deux parties :
\begin{itemize}
\item{un menu `fichier' permettant de charger ou de sauvegarder le contenu de l'arborescence.}
\item{un menu `outil' permettant d'ex\'ecuter des scripts de lecture de carte \`a puce.}
\end{itemize}

L'arborescence repr\'esente la structure de la carte \`a puce, telle qu'elle est lue par le script qui est ex\'ecut\'e. Cette arborescence est enti\`erement construite par le script (voir les fonctions de l'unit\'e 'ui' ci-apr\`es).

La console de message affiche des messages informatifs ou d'alerte cr\'e\'es par l'application ou le script en cours d'ex\'ecution. (voir la commande log.print ci-apr\`es).

\section{Descriptif du langage de script}

Les outils permettant de lire les diff\'erents formats de carte \`a puce sont stock\'es dans votre r\'epertoire \texttt{\$HOME/.cardpeek/scripts/}. 
Ces outils sont \'ecris en LUA, un langage de programmation qui ressemble au C et au Pascal (voir \texttt{http://www.lua.org/}). 
Pour permettre de dialoguer avec des cartes \`a puce, le langage LUA a \'et\'e \'etendu. 
Les paragraphes suivants pr\'esentent ces extensions.

\begin{center}
{\Large A compl\'eter.}
\end{center}

\section{Licence}

\texttt{Cardpeek} est un logiciel libre ; vous pouvez le redistribuer ou le modifier suivant les termes de la "GNU General Public License" telle que publi\'ee par la Free Software Foundation : soit la version 3 de cette licence, soit (\`a votre gr\'e) toute version ult\'erieure.
  
\texttt{Cardpeek} est distribu\'e dans l'espoir qu'il vous sera utile, mais SANS AUCUNE GARANTIE : sans m\^eme la garantie implicite de COMMERCIALISABILIT\'E ni d'AD\'EQUATION \'A UN OBJECTIF PARTICULIER. Consultez la Licence G\'en\'erale Publique GNU pour plus de d\'etails.
  
Vous devriez avoir re�u une copie de la Licence G\'en\'erale Publique GNU avec ce programme ; si ce n'est pas le cas, consultez :
\texttt{http://www.gnu.org/licenses/}.
\end{document}
