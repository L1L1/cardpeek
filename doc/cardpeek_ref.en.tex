\documentclass[11pt]{article}
\usepackage{url}
\title{\texttt{Cardpeek} Reference Manual - 0.2}
\author{Copyright 2009, L1L1@gmx.com}
\addtolength{\textwidth}{2cm}
\addtolength{\hoffset}{-1cm}
\begin{document}
\maketitle

\texttt{Cardpeek} is a program that reads the contents of smartcards. 
This GNU/Linux tool has a GTK GUI and can be extended with the LUA programming language. 
It requires a PCSC card-reader to communicate with a smartcard.

\section{Presentation}

Smartcards are becoming ubiquitous in our everyday life. 
We use them for payment, transport, in mobile telephones and many other applications.
These cards often contain a lot of personal information such as, for example, our last purchases or our last journeys in public transport.

\texttt{Cardpeek}'s goal is to allow you to access all this personal information. As such, you can be better informed about the data that is collected about you.

\texttt{Cardpeek} explores ISO 7816 compliant smartcards and represents their content in an organized tree format that roughly follows the structure it has inside the card, which is also similar to a classical file-system structure.

In this version, this tool is capable of reading the contents of the following types of cards:
\begin{itemize}
\item{EMV "chip and PIN" bank cards used in many countries throughout the world}
\item{Navigo transport cards used in Paris and other \textit{Calypso} cards used in France and other countries.}
\item{Moneo, the French electronic purse.}
\end{itemize}

Note that \textit{Calypso} card processing is still in `beta' state, 
and that some important card types are missing such as the SIM card used in mobile phones.
However, this application can be modified and extended easily to your needs with the embedded LUA scripting language.
For more information on the LUA project see \url{http://www.lua.org/}.

\section{Installation}

\texttt{Cardpeek} can be compiled from source using the provided \texttt{Makefile}.
It has currently only been tested under Linux Debian version 5.

\subsection{Compiling and installing}
Instructions:
\begin{enumerate}

\item{Make sure you have the following development packages installed:
  \begin{itemize}
  \item{libgtk 2.0 (\url{http://www.gtk.org)}}
  \item{liblua 5.1 (\url{http://www.lua.org)}}
  \item{libpcsclite (\url{http://pcsclite.alioth.debian.org/)}}
  \end{itemize}
(Note: On a Debian system, these packages are all available through \texttt{apt}/\texttt{aptitude}.)
}

\item{Unpack the source if needed and change directory to the source directory.}

\item{Type `\texttt{make}'}

\item{The binary executable is `\texttt{cardtree}', use it directly or copy it to the desired location.}

\end{enumerate}

\subsection{Related files and initial setup}
When first running \texttt{cardpeek} the program will attempt to create the
\texttt{.cardpeek/} directory in your home directory. This is normal.

The \texttt{.cardpeek/} directory will contain two elements: \texttt{config.lua} and the \texttt{script/} directory.
The \texttt{config.lua} allows you to run commands automatically when the program starts 
(it is currently only used for debugging purposes but should become a full fledged `config file' in the future).
The \texttt{script/} directory contains all the scripts that allow to explore smartcards. 
Currently it contains four LUA files: \texttt{emv.lua}, \texttt{navigo.lua}, \texttt{moneo.lua} and \texttt{atr.lua}.
These files all show up in the `analyzer' menu of the \texttt{cardpeek} (without their extension `.lua').
If you add any LUA file to this directory, it will thus also appear in the menu.
The \texttt{script/} contains a subdirectory \texttt{lib/} which holds a few LUA files that represent frequently used commands that are shared among the card processing scripts.

Each time the program runs, it creates a file \texttt{.cardpeek.log} in your home directory.
This file contains a copy of the messages displayed in the ``log'' tab of the application (see next section). 

\section{User interface description}

\subsection{Quick start}

To experiment with \texttt{cardpeek}, your may start with your EMV ``PIN and chip'' smartcard for example, by follow these steps:
\begin{enumerate}
\item{Start \texttt{cardpeek}.}
\item{Select your PCSC card reader in the first dialog box.}
\item{Select \emph{emv} in the \emph{analyzer} menu. This will run the default \emph{emv} script.}
\item{View the results in the ``card data'' tab.}
\end{enumerate}

\subsection{Details}

The user interface is divided in four parts: the top menu bar, the ``card data'' tab, the ``log'' tab and a one-line command input form.

The top menu bar holds 3 menus:
\begin{itemize}
\item{A \emph{data} menu that allows to clear, load or save data in the card tree browser.}  
\item{A \emph{analyzer} menu that allows to load a script or to run one of the predefined scripts.}
\item{A \emph{reader} menu that allows to connect, reset or disconnect a smartcard inserted in a card reader.}
\end{itemize}

The ``card data'' tab represents the data extracted from a card in a structured \emph{tree} from.
This tree structure is initially blank and is entirely constructed by the LUA scripts that are executed (see next section).
This tree can be saved and loaded in XML format using the commands in the \emph{data} menu.

The ``log'' tab keeps track of \emph{messages} emitted by the application or the script being run.
These messages are useful for monitoring and for debugging purposes. 
The last message also appears at the bottom of the screen in the status bar.

The one-line input form allows to type single LUA commands that will be directly executed by the application.
This is useful for testing some ideas quickly or debugging purposes.

When the program starts, you'll be asked to choose a card-reader.
This allows you to select which card-reader to use if you have several of them attached to your computer.
You may also select `\texttt{none}' if you do not wish to run any script on a card. This option is useful if you
only want to load and examine card data that was previously saved in XML format.

\section{Script language description}

The individual scripts that allow to process different types of smartcards are located in your \texttt{\$HOME/.cardpeek/scripts/} directory.
These scripts are written in LUA, a programming language which shares some similarities with Pascal and Javascript.
To allow LUA scripts to communicate with smartcards and to manipulate card data, the LUA language was extended with custom libraries.
This section first starts with a brief example of the script language and then describes the library extensions.

\subsection{Example}

Here's a short LUA script that demonstrates how to get and print the ATR (Answer To Reset) of a card to the console.
\begin{verbatim}
     card.connect()
     print(card.last_atr())
     card.disconnect()
\end{verbatim}

Next, here's (a slightly longer) script that performs a similar task, while presenting the results using \texttt{Cardpeek}'s user interface instead of the console. 

\begin{verbatim}
     card.connect()

     atr = card.last_atr()

     if atr then
        mycard = ui.tree_add_node(nil,"card")
        ref = ui.tree_add_node(mycard,"ATR","cold")
        ui.tree_set_value(ref,tostring(atr))
     end

     card.diconnect()
\end{verbatim}

The fist command \texttt{card.connect()} powers-up the card in the card reader and prepares the card for communication. 
Next \texttt{card.last\_atr()} returns the ATR of the card. 
If the value of the ATR is non-nil, the script creates a node called ``card'' (with \texttt{ui.tree\_add\_node()}). 
This node will appear at the root of the card data tree-view.
A child node called ``ATR'' is added to the root ``card'' node. 
The hexadecimal value of the ATR ---obtained by converting the \texttt{atr} variable to a string---  is associated with the child node.
Finaly, the card is powered down with the \texttt{card.disconnect()} function.

The final output of the script should have roughly the following structure:
\begin{verbatim}
   +-card
     |
     +-ATR cold     3B6900004A434F503331563232
\end{verbatim}

The example above is equivalent to the ``atr'' script provided with \texttt{Cardpeek}.
The LUA language is easy to learn and we refer the reader to \url{http://www.lua.org/} for more information.

\subsection{the \texttt{bit} library}

Since LUA does not have native bit manipulation functions, the following functions have been added.

\begin{description}

\item[\texttt{bit.and(A,B)}]
{Compute the binary operation \texttt{A} \textit{and} \texttt{B}.}

\item[\texttt{bit.or(A,B)}]
{Compute the binary operation \texttt{A} \textit{or} \texttt{B}.}

\item[\texttt{bit.xor(A,B)}]
{Compute the binary operation \texttt{A} \textit{xor} \texttt{B}.}

\item[\texttt{bit.shl(A,B)}]
{Compute $\texttt{A}{\times}2^{\texttt{B}}$.}

\item[\texttt{bit.shr(A,B)}]
{Compute $\texttt{A}/2^{\texttt{B}}$}

\end{description}

\subsection{The \texttt{bytes} library}

The \texttt{bytes} provides a new opaque type to LUA: a \textit{bytestring}, 
which is used to represent an array of binary elements. 

Bytestrings are mainly used to represent binary data exchanged with
the card reader in the application.

The elements in the array are most commonly (8 bit) bytes, but
it is also possible to construct arrays of (4 bit) half-bytes or
arrays of individual bits. The size of each element is specified 
when the array is created with the function \texttt{bytes.new()}
described in this section. A function to convert between different 
types of arrays is also provided.

Individual elements in a bytestring array can be accessed the same 
way traditional arrays are accessed in LUA. Thus, if \texttt{BS} is
a bytestring the following expressions are valid:
\begin{verbatim}
   BS[0]=1
   print(BS[0])
\end{verbatim} 
Contrary to the LUA tradition, the first index in a bytestring is 0
(instead of 1). The number of elements in a bytestring is indicated 
by prefixing the bytestring with the ``\texttt{\#}'' operator, just as 
with an array (e.g. \texttt{\#BS}).

Bytestrings cannot be copied like arrays with a simple assignment 
using the ``\texttt{=}'' operator, the \texttt{bytes.assign()} function
or the \texttt{bytes.clone()} must be used instead.  

The functions of the \texttt{bytes} library are:

\begin{description}

\item[\texttt{bytes.append(BS, $value_0$ [, $value_1$, ..., $value_n$])}]
{
  Append a value to $BS$.

  The appended value is composed of $value_0$, optionally concatenated with
  any additional value $value_1$, ..., $value_n$ (from left to right).

  This function is equivalent to 
  \texttt{bytes.assign(BS, BS, $value_0$ [, $value_1$, ..., $value_n$])}.
  See \texttt{bytes.assign()} for further details.

  This function modifies its main argument $BS$.

  This function returns \texttt{true} upon success and \texttt{false} otherwise.
}

\item[\texttt{bytes.assign(BS, $value_0$ [, $value_1$, ..., $value_n$])}]
{
  Assigns a value to $BS$. 

  The assigned value is composed of $value_0$, optionally concatenated with 
  any additional value $value_1$, ..., $value_n$ (from left to right).

  Each $value_i$ can be either a bytestring, a string or a number.
  If $value_i$ is a bytestring, each element of $value_i$ is appended to $BS$, 
  without any conversion.

  If $value_i$ is a string, it is interpreted as a text representation of
  a bytestring (as returned by the \texttt{tostring()} operator). This string
  representation is interpreted by taking into consideration the element width 
  of $BS$ and is appended to $BS$.

  If $value_i$ is a number, it converted into a single bytestring element and 
  appended to $BS$.

  This function modifies its main argument $BS$.

  This function returns \texttt{true} upon success and \texttt{false} otherwise.
} 

\item[\texttt{bytes.clone(BS)}]
{
  Creates and returns a copy of $BS$.
  This function returns \texttt{nil} if it fails.
}

\item[\texttt{bytes.concat($value_0$, $value_1$ [,$value_2$ , ..., $value_n$])}]
{
  Returns the concatenation of $value_0$, ..., $value_n$ (from left to right).

  For the rules governing the processing of $value_0 ... value_n$, see
  the \texttt{bytes.assign()} function above.

  This function returns a bytestring upon success and \texttt{nil} otherwise.
}

\item[\texttt{bytes.convert(w,BS)}]
{
  Converts $BS$ to a new bytestring where each element has a width $w$.

  Depending on $w$, the elements in the converted bytestring are obtained by either 
  splitting elements of $BS$ into severall smaller elements in the new bytestring or 
  by grouping several elements of $BS$ into a single element in the new bytestring.

  If the conversion requires splitting elements of $BS$, then the original elements will
  be split with the most significant bit(s) first (the most significant bits of each 
  original element of $BS$ will have a lower index than the least significant bits). 

  If the conversion requires grouping elements together, $BS$ is will first be 
  right-padded with zeros to a size that is a multiple of $w$. Next, new 
  elements are formed by considering elements of $BS$ with a lower index as more 
  significant than elements with a higher index.

  This function returns a new bytestring upon success and \texttt{nil} otherwise.
}

\item[\texttt{bytes.insert(BS, pos, $value_0$ [, $value_1$, ..., $value_n$])}]
{
  Inserts a value in $BS$ at index $pos$.

  The elements in $BS$ of index 0 to $pos-1$ will remain untouched.
  The elements in $BS$ of index $pos$ to \texttt{\#BS} are pushed to the right
  to make room for the inserted value.

  The inserted value is composed of $value_0$, optionally concatenated with
  any additional value $value_1$, ..., $value_n$ (from left to right).

  For the rules governing the processing of $value_0 ... value_n$, see
  the \texttt{bytes.assign()} function above.

  This function modifies its main argument $BS$.
  
  This function returns \texttt{true} upon success and \texttt{false} otherwise.
}

\item[\texttt{bytes.invert(BS)}]
{
  Reverses the order of elements in $BS$.

  If $BS$ has $N$ elements then $BS[0]$ is swapped with $BS[N-1]$,
  $BS[1]$ is swapped with $BS[N-2]$ and so forth until all elements are
  in reverse order in $BS$.

  This function modifies its main argument.

  This function returns \texttt{true} upon success and \texttt{false} otherwise.
}

\item[\texttt{bytes.is\_printable(BS)}]
{
  Returns \texttt{true} if all elements in $BS$ can be converted to 
  printable ascii characters, and \texttt{false} otherwise.

  This function always returns \texttt{false} if the width of $BS$ is 
  not 8 (elements of width 4 or 1 are not printable ascii values).  
}

\item[\texttt{bytes.maxn(BS)}]
{
  Returns the last index in $BS$ (equivalent to $\#BS-1$).

  This function returns \texttt{nil} if $BS$ is empty.
}

\item[\texttt{bytes.new(width [,$value_0$, $value_1$, ..., $value_n$])}]
{
  Creates a new bytestring, where each element is $width$ bits in size.
  $width$ can be either 8, 4 or 1.

  A value can optionally be assigned to the bytestring by specifying one or 
  several values $value_0$, $value_1$, ..., $value_n$ that will be concatenated
  together to form the content of the bytestring. See the function 
  \texttt{bytes.assign()} for more details.

  This function returns a bytestring upon success and \texttt{nil} otherwise. 
}

\item[Operators on bytestrings: ``\texttt{..}'', ``\texttt{==}'', ``\texttt{$\sim$=}'' and ``\texttt{\#}'']
{
  The ``..'' operator creates a new bytestring by concatenating two bytestrings together.
  The concatenation operator also works if one of the operands is a string or a number, 
  by converting it to a bytestring first, following the rules described in the 
  \texttt{bytes.assign()} function. 
  \texttt{A..B} is equivalent to \texttt{bytes.concat(A,B)}.
  
  The ``=='' and ``$\sim$='' operators allow to compare two bytestrings for equality or
  non-equality respectively. To be equal, two bytestrings must have the same width
  and the same elements in the same order.
 
  Finally the ``\#'' operator returns the number of elements in a bytestring.
}

\item[\texttt{bytes.pad\_left(BS, length, value)}]
{
  Pads $BS$ on the left with the element $value$ until the number of elements 
  of $BS$ reaches a multiple of $length$.

  If the size of $BS$ is already a multiple of $length$, $BS$ is left untouched.

  This function modifies its main argument $BS$.
  
  This function returns \texttt{true} upon success and \texttt{false} otherwise.
}

\item[\texttt{bytes.pad\_right(BS, length, value)}]
{
  Pads $BS$ on the right with the element $value$ until the number of elements 
  of $BS$ reaches a multiple of $length$.

  If the size of $BS$ is already a multiple of $length$, $BS$ is left untouched.

  This function modifies its main argument $BS$.
  
  This function returns \texttt{true} upon success and \texttt{false} otherwise.
}

\item[\texttt{bytes.remove(BS, start [,end])}]
{
  Deletes a part of $BS$.

  Removes all elements of $BS$ that have an index that verifies 
  $index \geq start$ and $index \leq end$.

  The elements in $BS$ are re-indexed: $BS[end+1]$ becomes $BS[start]$,
  $BS[end+2]$ becomes $BS[start+1]$, and so forth.

  If $end$ is not specified it will default to the last index of $BS$.
  $start$ and $end$ may be negative as described in \texttt{bytes.sub()}.

  This function modifies its main argument.
  
  This function returns \texttt{true} upon success and \texttt{false} otherwise.
}

\item[\texttt{bytes.sub(BS, start [,end])}]
{
  Returns a copy of a substring from $BS$.

  The returned value represents a bytestring containing a copy of all the 
  elements of $BS$ that have an index that verifies $index \geq start$ and 
  $index \leq end$.
  If $end$ is not specified it will default to the last index of $BS$.
  If $start$ (or $end$) is negative, it will be replaced by $\#BS+start$
  (or $\#BS+end$ resp.).

  This function returns a bytestring upon success and \texttt{nil} otherwise.
}

\item[\texttt{bytes.to\_number(BS)}]
{
  Converts the bytestring \texttt{BS} to a string representing the unsigned 
  decimal value of $BS$.

  This conversion considers $BS[0]$ as the most significant element of $BS$, 
  and $BS[\#BS-1]$ as the least significant. 
  
  If needed LUA will automatically convert the resulting string to a number.
}

\item[\texttt{bytes.to\_printable(BS)}]
{
  Converts each element in $BS$ into an ascii character and returns the
  resulting string.

  If an element in $BS$ cannot be converted to a printable character it
  is replaced by the character ``?''. 

  If $BS$ is empty, the resulting string is also empty.
}

\item[\texttt{bytes.width(BS)}]
{
  Return the width of the elements in $BS$.

  This function may return 1, 4 or 8.
}

\end{description}

\subsection{The \texttt{asn1} library}

The ASN1 library allows to manipulate ASN1 TLV bytestrings following the DER 
encoding (Distinguished Encoding Rules).
These bytestrings must be 8 bit wide.

The library provides the following functions:

\begin{description}

\item[\texttt{asn1.enable\_single\_byte\_length(enable)}]
{
  This function is only used in rare cases with erroneous card implementation.
  
  If $\texttt{enable}=true$ the behavior of TLV decoding functions 
  (such as \texttt{bytes.tlv\_split()}) are modified by forcing the ASN1 
  length to be 1 byte long. 
  
  This means that even if the first byte of the encoded length is greater 
  than $0x80$ it will be interpreted as the length of the TLV value.
}

\item[\texttt{asn1.join(tag, val [,extra])}]
{
  Performs the opposite of \texttt{bytes.tlv\_split}: creates a bytestring 
  representing the ASN1 DER encoding of the TLV $\{tag,len,val\}$ where $len=\#val$ 
  and appends \texttt{extra} to the result.

  \texttt{tag} is positive integer number, \texttt{val} is a bytestring and 
  \texttt{extra} is a bytestring or \texttt{nil}.
}

\item[\texttt{asn1.split(str)}]
{
  Parses the beginning of the bytestring \texttt{str} according to ASN1 BER 
  TLV encoding rules, and extracts a tag $T$ and a bytestring value $V$.

  The function then returns 3 elements $(T, V, extra)$, 
  where $extra$ is an optional bytestring representing the remaining part of 
  \texttt{str} that was not parsed or \texttt{nil} if no data remains.

  If this function fails it returns a triplet of \texttt{nil} values.
}	

\item[\texttt{asn1.split\_length(str)}]
{
  Parses the beginning of the bytestring \texttt{str} according to ASN1 BER 
  and extracts a length $L$. 

  The function then returns $(L, extra)$, where $extra$ is an optional bytestring 
  representing the remaining part of \texttt{str} that was not parsed or \texttt{nil} 
  if no data remains.

  If this function fails it returns a pair of \texttt{nil} values.
}

\item[\texttt{asn1.split\_tag(str)}]
{
  Parses the beginning of the bytestring \texttt{str} according to ASN1 BER 
  and extracts a tag $T$. 

  The function then returns $(T, extra)$, where $extra$ is an optional bytestring 
  representing the remaining part of \texttt{str} that was not parsed or 
  \texttt{nil} if no data remains.

  If this function fails it returns a pair of \texttt{nil} values.
}

\end{description}

\subsection{The \texttt{card} library}

The \texttt{card} library is used to communicate with a smartcard in a card-reader.

This library contains the following functions:

\begin{description}

\item[\texttt{card.connect()}]
{
  Connect to the card currently inserted in the selected smartcard reader. 

  This command is used at the start of most smartcard scripts.

  This function returns \texttt{true} on success, and \texttt{false} otherwise.
}

\item[\texttt{card.disconnect()}]
{
  Disconnect the card currently inserted in the selected smartcard reader.
	  
  This command concludes most smartcard scripts.

  This function returns \texttt{true} on success, and \texttt{false} otherwise.
}

\item[\texttt{card.last\_atr()}]
{
  Returns a bytestring representing the last ATR (Answer To Reset) returned by the card.
}

\item[\texttt{card.send(APDU)}]
{
  Sends the command \texttt{APDU} to the card. 

  The function returns a pair of values: a number representing the status word 
  returned by the card (ex. 0x9000) and the response data returned by the card.
  
  Both the command \texttt{APDU} and the response data are bytestrings 
  (see the \texttt{bytes} library).
}

\item[\texttt{card.status()}]
{
  \textit{Not implemented yet - will return detailed information about the state of the card reader.}
}

\item[\texttt{card.warm\_reset()}]
{
  Performs a warm reset of the card 
  (reconnects the card currently inserted in the selected smartcard reader).
}

\end{description}


\subsection{The \texttt{ui} library}

The \texttt{ui} library allows to control some elements of the user interface of 
\texttt{Cardpeek}, and in particular the tree structure representing the data 
extracted from the card.

The functions in this library rely on a value called a `\emph{path}' to identify 
each node in the card data tree represented on the screen.
A \emph{path} is a string that is composed of a list of positive numbers separated 
by a colon\footnote{For programmers familiar with GTK++ 2.0, these are the same 
conventions as in the GTK ``tree model''}. 
Each number represents the position (starting at 0) of a node relatively to its parent.
The \emph{path} ``0:2:1'' represents the second child node of the third child 
node of the first top node, and ``0'' simply represents the first top node.
The detailed value of a \emph{path} is usually not important for the programmer, 
who normally manipulates it as an opaque value. 

The \texttt{ui} library functions are the following:

\begin{description}

\item[\texttt{ui.tree\_add\_node(path\_to\_parent, name, [id, [length, [comment]]])}]
{
  Adds a node in the card tree structure. 

  The new node will be appended to the children of the node identified by the 
  \emph{path} \texttt{path\_to\_parent}.
  If \texttt{path\_to\_parent} is \texttt{nil} the new node will be added at 
  the top level.

  \texttt{name} describes the type of data that is represented by the node 
  (such as a ``file'' or a ``record'' for example). 

  \texttt{id} is an optional identifier that identifies the node uniquely within 
  a context (such as an application ``AID'' or a ``record number'').

  \texttt{length} is an optional value number describing the length of the 
  data element associated to the node.

  \texttt{comment} is an optional informative string that provides additional 
  information related to the node.
  
  Upon success the node returns a string representing the \emph{path} to the 
  newly created node. 
  If the function fails, it returns \texttt{nil}.
  Once the node is created with this function, data can be associated to it with the \texttt{ui.tree\_set\_value} function.
}

\item[\texttt{ui.tree\_set\_value(path\_to\_node, val)}]
{
  Associate the data \texttt{val} to the node identified by the \emph{path} 
  \texttt{path\_to\_node}.

  The value \texttt{val} is generally a bytestring, 
  as constructed by the \texttt{bytes} module functions.

  Once a value \texttt{val} is associated to a node \texttt{path\_to\_node} 
  it is not possible to add a child node with the function \texttt{ui.tree\_add\_node}.
	  
  The function returns \texttt{true} on success and \texttt{false} otherwise.
}

\item[\texttt{ui.tree\_get\_value(path\_to\_node)}]
{
  Returns the value associated with the node identified by \texttt{path\_to\_parent}, 
  or \texttt{nil} if no value is associated with the node or if the function fails.
} 

\item[\texttt{ui.tree\_get\_node(path\_to\_node)}]
{
  Returns an array of 5 elements associated to the node identified by 
  \texttt{path\_to\_node}.
 
  These elements are: ${name, id, length, comment, num\_children}$. 
  The first 4 elements are the same as the parameters of the function 
  \texttt{ui.tree\_add\_node}, and the last parameter describes the number of 
  children attached to that node in the tree (0 means none).

  If the function fails, it returns \texttt{nil}.
}

\item[\texttt{ui.tree\_delete\_node(path\_to\_node)}]
{
  Deletes the node identified by \texttt{path\_to\_node} as well as all its 
  children.
	  
  The function returns \texttt{true} on success and \texttt{false} otherwise.
}

\item[\texttt{ui.tree\_to\_xml(path\_to\_node)}]
{
  Returns an XML representation of the sub-tree that has \texttt{path\_to\_node} 
  as a root.
	  
  If \texttt{path\_to\_node} is \texttt{nil} the representation of the whole 
  tree is returned.
  
  If the function fails, it returns \texttt{nil}.
}

\item[\texttt{ui.tree\_find\_node(origin, name, id)}]
{
  Searches inside the sub-tree of root $origin$ for the first node that has 
  name \texttt{name} and/or the id \texttt{id}.

  If \texttt{name=nil} or \texttt{id=nil} they are ignored in the search.
  If a node is found, the function returns the path to that node otherwise 
  it returns \texttt{nil}.
}

\item[\texttt{ui.tree\_save(file\_name)}]
{
  Saves the tree in XML format inside the file $file\_name$.

  The function returns \texttt{true} on success and \texttt{false} otherwise.
}

\item[\texttt{ui.tree\_load(file\_name)}]
{
  Loads the tree from the XML file $file\_name$.

  The function returns \texttt{true} on success and \texttt{false} otherwise.
}

\item[\texttt{ui.question(text, buttons)}]
{
  Asks the user a question requesting him to answer by selecting a response.

  The question is described in the string \texttt{text}, while the set of 
  possible answers described in the table \texttt{buttons}. 
  Each element in \texttt{buttons} is string representing a possible answer.
  
  Upon success, the function returns the index of the answer selected by the 
  user in the table \texttt{buttons} 
  (LUA table indexes are usually numbers greater or equal to 1). 
  Upon failure failure the function returns 0.
}

\item[\texttt{ui.readline(text [,len [,default]])}]
{
  Request the user to enter a text string. 

  The user's input can optionally be limited to \texttt{len} characters and 
  can hold a predefined value \texttt{default}.
  
  The function returns the user's input upon success and \texttt{false} otherwise.
}

\end{description}

\subsection{The \texttt{log} library}

The \texttt{log} library contains just one function described below, which allows to print messages in the ``log'' tab of the application.

\begin{description}

\item[\texttt{log.print(level, text)}]
{
  Prints a message \texttt{text} in the console window.

  \texttt{level} describes the type of message that is printed. 
  \texttt{level} can take the following values: \texttt{log.INFO}, 
  \texttt{log.DEBUG}, \texttt{log.WARNING}, or \texttt{log.ERROR}.}

  All messages printed on the screen with this function are also 
  saved in the file ``\texttt{\$HOME/.cardpeek.log''.
}

\end{description}

\section{Future developments}

The short term development `road map' is:
\begin{description}
\item[GSM:]{Add a SIM card explorer script.}
\item[Navigo/Calypso:]{Provide better analysis of the card data, perhaps with city specific data.}
\item[Translation:]{Translate this document in French.}
\end{description}

\section{Licence}

\texttt{Cardpeek} is free software: you can redistribute it and/or modify
it under the terms of the GNU General Public License as published by
the Free Software Foundation, either version 3 of the License, or
(at your option) any later version.

\texttt{Cardpeek} is distributed in the hope that it will be useful,
but WITHOUT ANY WARRANTY; without even the implied warranty of
MERCHANTABILITY or FITNESS FOR A PARTICULAR PURPOSE.  See the
GNU General Public License for more details.

You should have received a copy of the GNU General Public License
along with this program.  If not, see \url{http://www.gnu.org/licenses/}.

\end{document}
